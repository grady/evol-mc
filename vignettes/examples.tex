\documentclass{article}\usepackage[]{graphicx}\usepackage[]{color}
%% maxwidth is the original width if it is less than linewidth
%% otherwise use linewidth (to make sure the graphics do not exceed the margin)
\makeatletter
\def\maxwidth{ %
  \ifdim\Gin@nat@width>\linewidth
    \linewidth
  \else
    \Gin@nat@width
  \fi
}
\makeatother

\definecolor{fgcolor}{rgb}{0.345, 0.345, 0.345}
\newcommand{\hlnum}[1]{\textcolor[rgb]{0.686,0.059,0.569}{#1}}%
\newcommand{\hlstr}[1]{\textcolor[rgb]{0.192,0.494,0.8}{#1}}%
\newcommand{\hlcom}[1]{\textcolor[rgb]{0.678,0.584,0.686}{\textit{#1}}}%
\newcommand{\hlopt}[1]{\textcolor[rgb]{0,0,0}{#1}}%
\newcommand{\hlstd}[1]{\textcolor[rgb]{0.345,0.345,0.345}{#1}}%
\newcommand{\hlkwa}[1]{\textcolor[rgb]{0.161,0.373,0.58}{\textbf{#1}}}%
\newcommand{\hlkwb}[1]{\textcolor[rgb]{0.69,0.353,0.396}{#1}}%
\newcommand{\hlkwc}[1]{\textcolor[rgb]{0.333,0.667,0.333}{#1}}%
\newcommand{\hlkwd}[1]{\textcolor[rgb]{0.737,0.353,0.396}{\textbf{#1}}}%

\usepackage{framed}
\makeatletter
\newenvironment{kframe}{%
 \def\at@end@of@kframe{}%
 \ifinner\ifhmode%
  \def\at@end@of@kframe{\end{minipage}}%
  \begin{minipage}{\columnwidth}%
 \fi\fi%
 \def\FrameCommand##1{\hskip\@totalleftmargin \hskip-\fboxsep
 \colorbox{shadecolor}{##1}\hskip-\fboxsep
     % There is no \\@totalrightmargin, so:
     \hskip-\linewidth \hskip-\@totalleftmargin \hskip\columnwidth}%
 \MakeFramed {\advance\hsize-\width
   \@totalleftmargin\z@ \linewidth\hsize
   \@setminipage}}%
 {\par\unskip\endMakeFramed%
 \at@end@of@kframe}
\makeatother

\definecolor{shadecolor}{rgb}{.97, .97, .97}
\definecolor{messagecolor}{rgb}{0, 0, 0}
\definecolor{warningcolor}{rgb}{1, 0, 1}
\definecolor{errorcolor}{rgb}{1, 0, 0}
\newenvironment{knitrout}{}{} % an empty environment to be redefined in TeX

\usepackage{alltt}
\usepackage{fullpage}
%\VignetteIndexEntry{evolMC demo} 
%\VignetteDepends{evolMC,knitr} 
%\VignetteEngine{knitr::knitr}




\title{evolMC demo}
\author{Grady Weyenberg}
\IfFileExists{upquote.sty}{\usepackage{upquote}}{}
\begin{document}
\maketitle

evolMC is a framework for doing Monte-Carlo simulations.

\subsection{Univariate and multivariate sampling}
We wish to use a metropolis sampler to draw from a distribution with
density \[ f(x) \propto \frac{\sin(x)}{x} \cdot 1_{(0,\pi)}(x). \] We
can use a uniform distribution on $(-1,1)$ to propose distances to
jump from the current location.


Since the proposal distribution is symmetric, this is enough
information to implement a Metropolis updater.

\begin{knitrout}
\definecolor{shadecolor}{rgb}{0.969, 0.969, 0.969}\color{fgcolor}\begin{kframe}
\begin{alltt}
\hlstd{fn} \hlkwb{<-} \hlkwa{function}\hlstd{(}\hlkwc{x}\hlstd{)} \hlkwd{sin}\hlstd{(x)}\hlopt{/}\hlstd{x} \hlopt{*} \hlstd{(}\hlnum{0} \hlopt{<} \hlstd{x)} \hlopt{*} \hlstd{(x} \hlopt{<} \hlstd{pi)}
\hlstd{propose} \hlkwb{<-} \hlkwa{function}\hlstd{(}\hlkwc{x}\hlstd{) x} \hlopt{+} \hlkwd{runif}\hlstd{(}\hlkwd{length}\hlstd{(x),} \hlopt{-}\hlnum{1}\hlstd{,} \hlnum{1}\hlstd{)}
\hlstd{updater} \hlkwb{<-} \hlkwd{metropolis}\hlstd{(fn, propose)}
\end{alltt}
\end{kframe}
\end{knitrout}


A Markov chain is formed by iteratively calling the updating function
starting with some initial value.
\begin{knitrout}
\definecolor{shadecolor}{rgb}{0.969, 0.969, 0.969}\color{fgcolor}\begin{kframe}
\begin{alltt}
\hlstd{chain} \hlkwb{<-} \hlkwd{iterate}\hlstd{(}\hlkwc{n} \hlstd{=} \hlnum{10000}\hlstd{,} \hlkwc{fn} \hlstd{= updater,} \hlkwc{init} \hlstd{=} \hlnum{1}\hlstd{)}
\hlkwd{summary}\hlstd{(chain)}
\end{alltt}
\begin{verbatim}
## Discarding first 1000 states.
##    mean      se    2.5%   97.5% 
## 1.11712 0.73918 0.04531 2.66390
\end{verbatim}
\begin{alltt}
\hlkwd{hist}\hlstd{(chain,} \hlkwc{breaks} \hlstd{=} \hlstr{"fd"}\hlstd{)}
\end{alltt}
\begin{verbatim}
## Discarding first 1000 states.
\end{verbatim}
\end{kframe}
\includegraphics[width=\maxwidth]{figure/unnamed-chunk-1} 

\end{knitrout}


Of course, multivariate distributions may also be sampled. 
\begin{knitrout}
\definecolor{shadecolor}{rgb}{0.969, 0.969, 0.969}\color{fgcolor}\begin{kframe}
\begin{alltt}
\hlstd{mvtarget} \hlkwb{<-} \hlkwa{function}\hlstd{(}\hlkwc{x}\hlstd{)} \hlkwd{prod}\hlstd{(}\hlkwd{sin}\hlstd{(}\hlkwd{prod}\hlstd{(x))}\hlopt{/}\hlstd{x)} \hlopt{*} \hlkwd{all}\hlstd{(x} \hlopt{>} \hlnum{0}\hlstd{, x} \hlopt{<} \hlstd{pi,} \hlkwd{prod}\hlstd{(x)} \hlopt{<}
    \hlstd{pi)}
\hlstd{mvupdate} \hlkwb{<-} \hlkwd{metropolis}\hlstd{(mvtarget, propose)}
\hlstd{chain2} \hlkwb{<-} \hlkwd{iterate}\hlstd{(}\hlnum{10000}\hlstd{, mvupdate,} \hlkwd{cbind}\hlstd{(}\hlkwc{x} \hlstd{=} \hlnum{1}\hlstd{,} \hlkwc{y} \hlstd{=} \hlnum{1}\hlstd{))}
\hlkwd{summary}\hlstd{(chain2)}
\end{alltt}
\begin{verbatim}
## Discarding first 1000 states.
##            x      y
## mean  1.2974 1.2333
## se    0.8146 0.8110
## 2.5%  0.1614 0.1618
## 97.5% 3.0129 2.9627
\end{verbatim}
\begin{alltt}
\hlkwd{plot}\hlstd{(chain2)}
\end{alltt}
\end{kframe}
\includegraphics[width=\maxwidth]{figure/mvt1} 
\begin{kframe}\begin{alltt}
\hlkwd{plot}\hlstd{(}\hlkwd{t}\hlstd{(}\hlkwd{simplify2array}\hlstd{(chain2)[}\hlnum{1}\hlstd{, , ]))}
\end{alltt}
\end{kframe}
\includegraphics[width=\maxwidth]{figure/mvt2} 

\end{knitrout}


\subsection{Multiple chains in parallel}
If the target density returns a vector instead of a scalar, we can run
multiple chains in parallel. (In this case the state object is a
matrix with individuals in rows.) In this situation, the proposal
updater should update the entire population.
\begin{knitrout}
\definecolor{shadecolor}{rgb}{0.969, 0.969, 0.969}\color{fgcolor}\begin{kframe}
\begin{alltt}
\hlstd{chain3} \hlkwb{<-} \hlkwd{iterate}\hlstd{(}\hlnum{100}\hlstd{, updater,} \hlkwc{init} \hlstd{=} \hlkwd{rbind}\hlstd{(}\hlnum{0.1}\hlstd{,} \hlnum{0.5}\hlstd{,} \hlnum{2.5}\hlstd{,} \hlnum{3.1}\hlstd{))}
\hlkwd{plot}\hlstd{(chain3)}
\end{alltt}
\end{kframe}
\includegraphics[width=\maxwidth]{figure/parallel} 

\end{knitrout}


Multiple chains of multivariate samples are also possible, as long as
a logical vector of the same length as that returned by the density
function will ``correctly'' subset individuals from the population
state object. \footnote{Thus, a population can be a vector, a matrix with
individuals in rows, or a list. Some convenience methods such as
print, summary, plot, hist, etc., assume a population is a matrix, but
this is not strictly required.}
\begin{knitrout}
\definecolor{shadecolor}{rgb}{0.969, 0.969, 0.969}\color{fgcolor}\begin{kframe}
\begin{alltt}
\hlstd{mvt2} \hlkwb{<-} \hlkwa{function}\hlstd{(}\hlkwc{x}\hlstd{)} \hlkwd{apply}\hlstd{(x,} \hlnum{1}\hlstd{, mvtarget)}
\hlstd{mvup2} \hlkwb{<-} \hlkwd{metropolis}\hlstd{(mvt2, propose)}
\hlstd{init} \hlkwb{<-} \hlkwd{rbind}\hlstd{(}\hlkwd{c}\hlstd{(}\hlkwc{x} \hlstd{=} \hlnum{1}\hlstd{,} \hlkwc{y} \hlstd{=} \hlnum{3}\hlstd{),} \hlkwd{c}\hlstd{(}\hlnum{0.1}\hlstd{,} \hlnum{0.5}\hlstd{),} \hlkwd{c}\hlstd{(}\hlnum{1}\hlstd{,} \hlnum{1}\hlstd{))}
\hlstd{chain4} \hlkwb{<-} \hlkwd{iterate}\hlstd{(}\hlnum{1000}\hlstd{, mvup2, init)}
\hlkwd{plot}\hlstd{(chain4)}
\end{alltt}
\end{kframe}
\includegraphics[width=\maxwidth]{figure/multi-multi} 

\end{knitrout}


\subsection{Gibbs sampling}
\label{sec:gibbs-sampling}
A gibbs updater calls several updating functions sequentially. Here we
find Baysean location-scale parameter estimates for a t-distribution
likelihood with known degrees of freedom $5/2$. The model
specification in this case is $y|\mu,w\sim N(\mu,w^{-1})$ and
$w|\sigma^2,\nu\sim G(\nu/2,\sigma^2\nu/2)$, with priors $\mu\sim N$
and $\sigma^2\sim G$.
\begin{knitrout}
\definecolor{shadecolor}{rgb}{0.969, 0.969, 0.969}\color{fgcolor}\begin{kframe}
\begin{alltt}
\hlstd{newcomb} \hlkwb{<-} \hlkwd{c}\hlstd{(}\hlnum{28}\hlstd{,} \hlopt{-}\hlnum{44}\hlstd{,} \hlnum{29}\hlstd{,} \hlnum{30}\hlstd{,} \hlnum{26}\hlstd{,} \hlnum{27}\hlstd{,} \hlnum{22}\hlstd{,} \hlnum{23}\hlstd{,} \hlnum{33}\hlstd{,} \hlnum{16}\hlstd{,} \hlnum{24}\hlstd{,} \hlnum{29}\hlstd{,} \hlnum{24}\hlstd{,} \hlnum{40}\hlstd{,} \hlnum{21}\hlstd{,} \hlnum{31}\hlstd{,}
    \hlnum{34}\hlstd{,} \hlopt{-}\hlnum{2}\hlstd{,} \hlnum{25}\hlstd{,} \hlnum{19}\hlstd{,} \hlnum{24}\hlstd{,} \hlnum{28}\hlstd{,} \hlnum{37}\hlstd{,} \hlnum{32}\hlstd{,} \hlnum{20}\hlstd{,} \hlnum{25}\hlstd{,} \hlnum{25}\hlstd{,} \hlnum{36}\hlstd{,} \hlnum{36}\hlstd{,} \hlnum{21}\hlstd{,} \hlnum{28}\hlstd{,} \hlnum{26}\hlstd{,} \hlnum{32}\hlstd{,} \hlnum{28}\hlstd{,}
    \hlnum{26}\hlstd{,} \hlnum{30}\hlstd{,} \hlnum{36}\hlstd{,} \hlnum{29}\hlstd{,} \hlnum{30}\hlstd{,} \hlnum{22}\hlstd{,} \hlnum{36}\hlstd{,} \hlnum{27}\hlstd{,} \hlnum{26}\hlstd{,} \hlnum{28}\hlstd{,} \hlnum{29}\hlstd{,} \hlnum{23}\hlstd{,} \hlnum{31}\hlstd{,} \hlnum{32}\hlstd{,} \hlnum{24}\hlstd{,} \hlnum{27}\hlstd{,} \hlnum{27}\hlstd{,} \hlnum{27}\hlstd{,}
    \hlnum{32}\hlstd{,} \hlnum{25}\hlstd{,} \hlnum{28}\hlstd{,} \hlnum{27}\hlstd{,} \hlnum{26}\hlstd{,} \hlnum{24}\hlstd{,} \hlnum{32}\hlstd{,} \hlnum{29}\hlstd{,} \hlnum{28}\hlstd{,} \hlnum{33}\hlstd{,} \hlnum{39}\hlstd{,} \hlnum{25}\hlstd{,} \hlnum{16}\hlstd{,} \hlnum{23}\hlstd{)}
\hlcom{##' Full conditional updater for mu }
\hlstd{f.m} \hlkwb{<-} \hlkwa{function}\hlstd{(}\hlkwc{state}\hlstd{,} \hlkwc{...}\hlstd{) \{}
    \hlstd{k} \hlkwb{<-} \hlkwd{seq_len}\hlstd{(}\hlkwd{ncol}\hlstd{(state)} \hlopt{-} \hlkwd{length}\hlstd{(newcomb))}
    \hlstd{w} \hlkwb{<-} \hlkwd{rowSums}\hlstd{(state[,} \hlopt{-}\hlstd{k])}
    \hlstd{mu} \hlkwb{<-} \hlstd{(state[,} \hlopt{-}\hlstd{k]} \hlopt \hlstd{newcomb)}\hlopt{/}\hlstd{(}\hlnum{1e-04} \hlopt{+} \hlstd{w)}
    \hlstd{sd} \hlkwb{<-} \hlnum{1}\hlopt{/}\hlkwd{sqrt}\hlstd{(}\hlnum{1e-04} \hlopt{+} \hlstd{w)}
    \hlstd{n} \hlkwb{<-} \hlkwd{nrow}\hlstd{(state)}
    \hlstd{state[,} \hlnum{1}\hlstd{]} \hlkwb{<-} \hlkwd{rnorm}\hlstd{(n, mu, sd)}
    \hlstd{state}
\hlstd{\}}
\hlcom{##' Full conditional updater for sigma^2}
\hlstd{f.v} \hlkwb{<-} \hlkwa{function}\hlstd{(}\hlkwc{state}\hlstd{,} \hlkwc{df} \hlstd{=} \hlnum{FALSE}\hlstd{) \{}
    \hlstd{nu} \hlkwb{<-} \hlkwa{if} \hlstd{(df)}
        \hlnum{1}\hlopt{/}\hlstd{state[,} \hlnum{3}\hlstd{]} \hlkwa{else} \hlnum{5}
    \hlstd{k} \hlkwb{<-} \hlkwd{seq_len}\hlstd{(}\hlkwd{ncol}\hlstd{(state)} \hlopt{-} \hlkwd{length}\hlstd{(newcomb))}
    \hlstd{n} \hlkwb{<-} \hlkwd{ncol}\hlstd{(state[,} \hlopt{-}\hlstd{k])}
    \hlstd{rates} \hlkwb{<-} \hlnum{0.1} \hlopt{+} \hlkwd{rowSums}\hlstd{(state[,} \hlopt{-}\hlstd{k])} \hlopt{*} \hlstd{nu}\hlopt{/}\hlnum{2}
    \hlstd{state[,} \hlnum{2}\hlstd{]} \hlkwb{<-} \hlkwd{rgamma}\hlstd{(}\hlkwd{nrow}\hlstd{(state),} \hlnum{0.1} \hlopt{+} \hlstd{n} \hlopt{*} \hlstd{nu}\hlopt{/}\hlnum{2}\hlstd{, rates)}
    \hlstd{state}
\hlstd{\}}
\hlcom{##' Full conditional updater for w}
\hlstd{f.w} \hlkwb{<-} \hlkwa{function}\hlstd{(}\hlkwc{state}\hlstd{,} \hlkwc{df} \hlstd{=} \hlnum{FALSE}\hlstd{) \{}
    \hlstd{nu} \hlkwb{<-} \hlkwa{if} \hlstd{(df)}
        \hlnum{1}\hlopt{/}\hlstd{state[,} \hlnum{3}\hlstd{]} \hlkwa{else} \hlnum{5}
    \hlstd{k} \hlkwb{<-} \hlkwd{seq_len}\hlstd{(}\hlkwd{ncol}\hlstd{(state)} \hlopt{-} \hlkwd{length}\hlstd{(newcomb))}
    \hlstd{rate} \hlkwb{<-} \hlstd{(nu} \hlopt{*} \hlstd{state[,} \hlnum{2}\hlstd{]} \hlopt{+} \hlkwd{t}\hlstd{(}\hlkwd{apply}\hlstd{(state,} \hlnum{1}\hlstd{,} \hlkwa{function}\hlstd{(}\hlkwc{row}\hlstd{) (newcomb} \hlopt{-} \hlstd{row[}\hlnum{1}\hlstd{])}\hlopt{^}\hlnum{2}\hlstd{)))}\hlopt{/}\hlnum{2}
    \hlstd{state[,} \hlopt{-}\hlstd{k]} \hlkwb{<-} \hlkwd{rgamma}\hlstd{(}\hlkwd{length}\hlstd{(state[,} \hlopt{-}\hlstd{k]), (nu} \hlopt{+} \hlnum{1}\hlstd{)}\hlopt{/}\hlnum{2}\hlstd{, rate)}
    \hlstd{state}
\hlstd{\}}
\end{alltt}
\end{kframe}
\end{knitrout}

The gibbs function is used to create a function which acts as the
Gibbs updater.

\begin{knitrout}
\definecolor{shadecolor}{rgb}{0.969, 0.969, 0.969}\color{fgcolor}\begin{kframe}
\begin{alltt}
\hlstd{init} \hlkwb{<-} \hlkwd{rbind}\hlstd{(}\hlkwd{c}\hlstd{(}\hlkwc{mean} \hlstd{=} \hlnum{25}\hlstd{,} \hlkwc{var} \hlstd{=} \hlnum{22}\hlstd{,} \hlkwd{rgamma}\hlstd{(}\hlkwd{length}\hlstd{(newcomb),} \hlnum{5}\hlopt{/}\hlnum{2}\hlstd{,} \hlnum{22} \hlopt{*} \hlnum{5}\hlopt{/}\hlnum{2}\hlstd{)),}
    \hlkwd{c}\hlstd{(}\hlnum{10}\hlstd{,} \hlnum{10}\hlstd{,} \hlkwd{rgamma}\hlstd{(}\hlkwd{length}\hlstd{(newcomb),} \hlnum{5}\hlopt{/}\hlnum{2}\hlstd{,} \hlnum{10} \hlopt{*} \hlnum{5}\hlopt{/}\hlnum{2}\hlstd{)))}
\hlstd{newcomb.gibbs} \hlkwb{<-} \hlkwd{gibbs}\hlstd{(f.m, f.v, f.w)}
\hlstd{chain5} \hlkwb{<-} \hlkwd{iterate}\hlstd{(}\hlnum{10000}\hlstd{, newcomb.gibbs, init)}
\hlstd{chain5} \hlkwb{<-} \hlkwd{prune}\hlstd{(chain5,} \hlnum{TRUE}\hlstd{,} \hlnum{1}\hlopt{:}\hlnum{2}\hlstd{)}
\hlkwd{plot}\hlstd{(chain5)}
\end{alltt}
\end{kframe}
\includegraphics[width=\maxwidth]{figure/gibbs} 
\begin{kframe}\begin{alltt}
\hlkwd{summary}\hlstd{(chain5)}
\end{alltt}
\begin{verbatim}
## Discarding first 1000 states.
##          mean    var
## mean  27.5091 21.219
## se     0.6419  4.511
## 2.5%  26.2784 13.794
## 97.5% 28.7828 31.273
\end{verbatim}
\end{kframe}
\end{knitrout}


Of course, you can use a metropolis updater in place of a full
conditional. Here we put a prior on $\nu^{-1} \sim E(1)T(0,1)$
\begin{knitrout}
\definecolor{shadecolor}{rgb}{0.969, 0.969, 0.969}\color{fgcolor}\begin{kframe}
\begin{alltt}
\hlstd{dtexp} \hlkwb{<-} \hlkwa{function}\hlstd{(}\hlkwc{x}\hlstd{,} \hlkwc{rate} \hlstd{=} \hlnum{1}\hlstd{,} \hlkwc{min} \hlstd{=} \hlnum{0}\hlstd{,} \hlkwc{max} \hlstd{=} \hlnum{Inf}\hlstd{,} \hlkwc{log} \hlstd{=} \hlnum{FALSE}\hlstd{) \{}
    \hlstd{d} \hlkwb{<-} \hlkwd{dexp}\hlstd{(x, rate)} \hlopt{*} \hlstd{(x} \hlopt{>=} \hlstd{min)} \hlopt{*} \hlstd{(x} \hlopt{<=} \hlstd{max)}
    \hlstd{c} \hlkwb{<-} \hlkwd{pexp}\hlstd{(max, rate)} \hlopt{-} \hlkwd{pexp}\hlstd{(min, rate)}
    \hlkwa{if} \hlstd{(log)}
        \hlkwd{log}\hlstd{(d}\hlopt{/}\hlstd{c)} \hlkwa{else} \hlstd{d}\hlopt{/}\hlstd{c}
\hlstd{\}}
\hlstd{posterior} \hlkwb{<-} \hlkwa{function}\hlstd{(}\hlkwc{state}\hlstd{) \{}
    \hlkwa{if} \hlstd{(}\hlopt{!}\hlkwd{is.matrix}\hlstd{(state))}
        \hlstd{state} \hlkwb{<-} \hlkwd{matrix}\hlstd{(state,} \hlnum{1}\hlstd{)}
    \hlstd{f} \hlkwb{<-} \hlkwa{function}\hlstd{(}\hlkwc{x}\hlstd{)} \hlkwd{sum}\hlstd{(}\hlkwd{dt}\hlstd{((newcomb} \hlopt{-} \hlstd{x[}\hlnum{1}\hlstd{])}\hlopt{/}\hlkwd{sqrt}\hlstd{(x[}\hlnum{2}\hlstd{]),} \hlnum{1}\hlopt{/}\hlstd{x[}\hlnum{3}\hlstd{],} \hlkwc{log} \hlstd{=} \hlnum{TRUE}\hlstd{)} \hlopt{-}
        \hlkwd{log}\hlstd{(x[}\hlnum{2}\hlstd{])}\hlopt{/}\hlnum{2}\hlstd{)}
    \hlkwd{exp}\hlstd{(}\hlkwd{apply}\hlstd{(state,} \hlnum{1}\hlstd{, f)} \hlopt{+} \hlkwd{dnorm}\hlstd{(state[,} \hlnum{1}\hlstd{],} \hlkwc{sd} \hlstd{=} \hlnum{100}\hlstd{,} \hlkwc{log} \hlstd{=} \hlnum{TRUE}\hlstd{)} \hlopt{+} \hlkwd{dgamma}\hlstd{(state[,}
        \hlnum{2}\hlstd{],} \hlnum{0.1}\hlstd{,} \hlnum{0.1}\hlstd{,} \hlkwc{log} \hlstd{=} \hlnum{TRUE}\hlstd{)} \hlopt{+} \hlkwd{dtexp}\hlstd{(state[,} \hlnum{3}\hlstd{],} \hlkwc{max} \hlstd{=} \hlnum{1}\hlstd{,} \hlkwc{log} \hlstd{=} \hlnum{TRUE}\hlstd{))}
\hlstd{\}}
\hlstd{kappa.prop} \hlkwb{<-} \hlkwa{function}\hlstd{(}\hlkwc{state}\hlstd{) \{}
    \hlstd{state[,} \hlnum{3}\hlstd{]} \hlkwb{<-} \hlkwd{runif}\hlstd{(}\hlkwd{nrow}\hlstd{(state))}
    \hlstd{state}
\hlstd{\}}
\hlstd{init} \hlkwb{<-} \hlkwd{rbind}\hlstd{(}\hlkwd{c}\hlstd{(}\hlkwc{mean} \hlstd{=} \hlnum{25}\hlstd{,} \hlkwc{var} \hlstd{=} \hlnum{22}\hlstd{,} \hlkwc{kappa} \hlstd{=} \hlnum{0.5}\hlstd{,} \hlkwd{rgamma}\hlstd{(}\hlkwd{length}\hlstd{(newcomb),} \hlnum{5}\hlopt{/}\hlnum{2}\hlstd{,}
    \hlnum{22} \hlopt{*} \hlnum{5}\hlopt{/}\hlnum{2}\hlstd{)),} \hlkwd{c}\hlstd{(}\hlnum{10}\hlstd{,} \hlnum{10}\hlstd{,} \hlnum{0.4}\hlstd{,} \hlkwd{rgamma}\hlstd{(}\hlkwd{length}\hlstd{(newcomb),} \hlnum{5}\hlopt{/}\hlnum{2}\hlstd{,} \hlnum{10} \hlopt{*} \hlnum{5}\hlopt{/}\hlnum{2}\hlstd{)))}
\hlstd{newcomb.chain3} \hlkwb{<-} \hlkwd{gibbs}\hlstd{(f.m, f.v, f.w,} \hlkwd{metropolis}\hlstd{(posterior, kappa.prop))}
\hlstd{chain4} \hlkwb{<-} \hlkwd{iterate}\hlstd{(}\hlnum{10000}\hlstd{, newcomb.chain3, init,} \hlkwc{df} \hlstd{=} \hlnum{TRUE}\hlstd{)}
\hlstd{chain4} \hlkwb{<-} \hlkwd{prune}\hlstd{(chain4,} \hlnum{TRUE}\hlstd{,} \hlnum{1}\hlopt{:}\hlnum{3}\hlstd{)}
\hlkwd{summary}\hlstd{(chain4)}
\end{alltt}
\begin{verbatim}
## Discarding first 1000 states.
##          mean    var  kappa
## mean  27.3737 12.950 0.5509
## se     0.5899  4.063 0.1495
## 2.5%  26.2344  6.557 0.3040
## 97.5% 28.5677 22.392 0.8824
\end{verbatim}
\begin{alltt}
\hlkwd{plot}\hlstd{(chain4)}
\end{alltt}
\end{kframe}
\includegraphics[width=\maxwidth]{figure/gibbs-metropolis1} 
\begin{kframe}\begin{alltt}
\hlkwd{hist}\hlstd{(chain4,} \hlkwc{breaks} \hlstd{=} \hlstr{"fd"}\hlstd{)}
\end{alltt}
\begin{verbatim}
## Discarding first 1000 states.
\end{verbatim}
\end{kframe}
\includegraphics[width=\maxwidth]{figure/gibbs-metropolis2} 

\end{knitrout}


\subsection{Parallel tempering}
We want to sample from a distribution with separated modes.  First we
set up the target distribution function. This function should take a
single individual and return the density of that individual.
\begin{knitrout}
\definecolor{shadecolor}{rgb}{0.969, 0.969, 0.969}\color{fgcolor}\begin{kframe}
\begin{alltt}
\hlstd{f} \hlkwb{<-} \hlkwa{function}\hlstd{(}\hlkwc{state}\hlstd{)} \hlkwd{exp}\hlstd{(}\hlkwd{sum}\hlstd{(}\hlkwd{log}\hlstd{(}\hlkwd{dnorm}\hlstd{(state)} \hlopt{+} \hlkwd{dnorm}\hlstd{(state,} \hlnum{10}\hlstd{))))}
\end{alltt}
\end{kframe}
\end{knitrout}

Now we can set up the parallel tempering updater. Function temper is similar to
metropolis, but you must also provide a list of temperatures. (It is
assumed that the number of individuals in the population will be the
same as the number of temperature levels.)
\begin{knitrout}
\definecolor{shadecolor}{rgb}{0.969, 0.969, 0.969}\color{fgcolor}\begin{kframe}
\begin{alltt}
\hlstd{temps} \hlkwb{<-} \hlkwd{c}\hlstd{(}\hlnum{1}\hlstd{,} \hlnum{2}\hlstd{,} \hlnum{4}\hlstd{,} \hlnum{8}\hlstd{,} \hlnum{15}\hlstd{)}
\hlstd{updater} \hlkwb{<-} \hlkwd{temper}\hlstd{(f, temps, propose)}
\hlstd{init} \hlkwb{<-} \hlkwd{rbind}\hlstd{(}\hlopt{-}\hlnum{3}\hlstd{,} \hlnum{0}\hlstd{,} \hlnum{5}\hlstd{,} \hlnum{10}\hlstd{,} \hlnum{15}\hlstd{)}
\hlstd{gch} \hlkwb{<-} \hlkwd{iterate}\hlstd{(}\hlnum{10000}\hlstd{, updater, init)}
\hlkwd{plot}\hlstd{(gch)}
\end{alltt}
\end{kframe}
\includegraphics[width=\maxwidth]{figure/unnamed-chunk-41} 
\begin{kframe}\begin{alltt}
\hlstd{xx} \hlkwb{<-} \hlkwd{prune}\hlstd{(gch,} \hlnum{1}\hlstd{,} \hlnum{1}\hlstd{)}  \hlcom{#extract chain with correct distribution}
\hlkwd{hist}\hlstd{(xx,} \hlkwc{freq} \hlstd{=} \hlnum{FALSE}\hlstd{,} \hlkwc{breaks} \hlstd{=} \hlstr{"fd"}\hlstd{)}
\end{alltt}
\begin{verbatim}
## Discarding first 1000 states.
\end{verbatim}
\begin{alltt}
\hlkwd{curve}\hlstd{((}\hlkwd{dnorm}\hlstd{(x)} \hlopt{+} \hlkwd{dnorm}\hlstd{(x,} \hlnum{10}\hlstd{))}\hlopt{/}\hlnum{2}\hlstd{,} \hlkwc{add} \hlstd{=} \hlnum{TRUE}\hlstd{)}
\end{alltt}
\end{kframe}
\includegraphics[width=\maxwidth]{figure/unnamed-chunk-42} 

\end{knitrout}


\subsection{Evolutionary MC}

Here we go with the real-deal. Twenty part trivariate-gaussian mixure.
The target function is a part of the evolMC package. 

\begin{knitrout}
\definecolor{shadecolor}{rgb}{0.969, 0.969, 0.969}\color{fgcolor}\begin{kframe}
\begin{alltt}
\hlstd{init} \hlkwb{<-} \hlkwd{matrix}\hlstd{(}\hlnum{0}\hlstd{,} \hlnum{5}\hlstd{,} \hlnum{3}\hlstd{)}
\hlstd{means} \hlkwb{<-} \hlkwd{t}\hlstd{(}\hlkwd{replicate}\hlstd{(}\hlnum{20}\hlstd{,} \hlkwd{runif}\hlstd{(}\hlnum{3}\hlstd{,} \hlopt{-}\hlnum{10}\hlstd{,} \hlnum{10}\hlstd{)))}
\hlstd{ff} \hlkwb{<-} \hlkwa{function}\hlstd{(}\hlkwc{x}\hlstd{)} \hlkwd{target}\hlstd{(x, means)}
\hlstd{e} \hlkwb{<-} \hlkwd{seq}\hlstd{(}\hlnum{0.2}\hlstd{,} \hlnum{3}\hlstd{,} \hlkwc{length} \hlstd{=} \hlnum{5}\hlstd{)}
\hlstd{pf} \hlkwb{<-} \hlkwa{function}\hlstd{(}\hlkwc{x}\hlstd{) x} \hlopt{+} \hlkwd{runif}\hlstd{(}\hlkwd{length}\hlstd{(x),} \hlopt{-}\hlstd{e, e)}
\hlstd{chainz} \hlkwb{<-} \hlkwd{iterate}\hlstd{(}\hlnum{10000}\hlstd{,} \hlkwd{gibbs}\hlstd{(}\hlkwd{temper}\hlstd{(ff, temps, pf),} \hlkwd{crossover}\hlstd{(ff, temps)),}
    \hlstd{init)}
\end{alltt}
\end{kframe}
\end{knitrout}

\begin{knitrout}
\definecolor{shadecolor}{rgb}{0.969, 0.969, 0.969}\color{fgcolor}\begin{kframe}
\begin{alltt}
\hlkwd{plot}\hlstd{(chainz)}
\end{alltt}
\end{kframe}
\includegraphics[width=\maxwidth]{figure/unnamed-chunk-61} 
\begin{kframe}\begin{alltt}
\hlcom{## means hist(chainz)}
\hlstd{m} \hlkwb{<-} \hlkwd{aperm}\hlstd{(}\hlkwd{simplify2array}\hlstd{(chainz))}
\hlkwd{library}\hlstd{(scatterplot3d)}
\hlstd{pp} \hlkwb{<-} \hlkwd{scatterplot3d}\hlstd{(means)}
\hlkwd{sapply}\hlstd{(}\hlnum{5}\hlopt{:}\hlnum{1}\hlstd{,} \hlkwa{function}\hlstd{(}\hlkwc{i}\hlstd{) pp}\hlopt{$}\hlkwd{points3d}\hlstd{(m[, , i],} \hlkwc{col} \hlstd{= i} \hlopt{+} \hlnum{1}\hlstd{))}
\end{alltt}
\begin{verbatim}
## [[1]]
## NULL
## 
## [[2]]
## NULL
## 
## [[3]]
## NULL
## 
## [[4]]
## NULL
## 
## [[5]]
## NULL
\end{verbatim}
\begin{alltt}
\hlstd{pp}\hlopt{$}\hlkwd{points3d}\hlstd{(means,} \hlkwc{col} \hlstd{=} \hlnum{1}\hlstd{,} \hlkwc{pch} \hlstd{=} \hlnum{19}\hlstd{)}
\end{alltt}
\end{kframe}
\includegraphics[width=\maxwidth]{figure/unnamed-chunk-62} 
\begin{kframe}\begin{alltt}
## library(rgl) plot3d(m[,,1]) plot3d(means,pch=19,size=15)
## sapply(1:5,function(i) points3d(m[,,i],col=i+1))
\end{alltt}
\end{kframe}
\end{knitrout}

\end{document}
