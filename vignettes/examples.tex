\documentclass{article}\usepackage[]{graphicx}\usepackage[]{color}
%% maxwidth is the original width if it is less than linewidth
%% otherwise use linewidth (to make sure the graphics do not exceed the margin)
\makeatletter
\def\maxwidth{ %
  \ifdim\Gin@nat@width>\linewidth
    \linewidth
  \else
    \Gin@nat@width
  \fi
}
\makeatother

\definecolor{fgcolor}{rgb}{0.345, 0.345, 0.345}
\newcommand{\hlnum}[1]{\textcolor[rgb]{0.686,0.059,0.569}{#1}}%
\newcommand{\hlstr}[1]{\textcolor[rgb]{0.192,0.494,0.8}{#1}}%
\newcommand{\hlcom}[1]{\textcolor[rgb]{0.678,0.584,0.686}{\textit{#1}}}%
\newcommand{\hlopt}[1]{\textcolor[rgb]{0,0,0}{#1}}%
\newcommand{\hlstd}[1]{\textcolor[rgb]{0.345,0.345,0.345}{#1}}%
\newcommand{\hlkwa}[1]{\textcolor[rgb]{0.161,0.373,0.58}{\textbf{#1}}}%
\newcommand{\hlkwb}[1]{\textcolor[rgb]{0.69,0.353,0.396}{#1}}%
\newcommand{\hlkwc}[1]{\textcolor[rgb]{0.333,0.667,0.333}{#1}}%
\newcommand{\hlkwd}[1]{\textcolor[rgb]{0.737,0.353,0.396}{\textbf{#1}}}%

\usepackage{framed}
\makeatletter
\newenvironment{kframe}{%
 \def\at@end@of@kframe{}%
 \ifinner\ifhmode%
  \def\at@end@of@kframe{\end{minipage}}%
  \begin{minipage}{\columnwidth}%
 \fi\fi%
 \def\FrameCommand##1{\hskip\@totalleftmargin \hskip-\fboxsep
 \colorbox{shadecolor}{##1}\hskip-\fboxsep
     % There is no \\@totalrightmargin, so:
     \hskip-\linewidth \hskip-\@totalleftmargin \hskip\columnwidth}%
 \MakeFramed {\advance\hsize-\width
   \@totalleftmargin\z@ \linewidth\hsize
   \@setminipage}}%
 {\par\unskip\endMakeFramed%
 \at@end@of@kframe}
\makeatother

\definecolor{shadecolor}{rgb}{.97, .97, .97}
\definecolor{messagecolor}{rgb}{0, 0, 0}
\definecolor{warningcolor}{rgb}{1, 0, 1}
\definecolor{errorcolor}{rgb}{1, 0, 0}
\newenvironment{knitrout}{}{} % an empty environment to be redefined in TeX

\usepackage{alltt}
\usepackage{fullpage}
\usepackage{hyperref}
\usepackage{amsmath}
%\VignetteIndexEntry{evolMC demo} 
%\VignetteDepends{evolMC,knitr} 
%\VignetteEngine{knitr::knitr}




\title{evolMC demo}
\author{Grady Weyenberg}
\IfFileExists{upquote.sty}{\usepackage{upquote}}{}
\begin{document}
\maketitle
\setcounter{section}{1}
evolMC is a framework for doing Monte-Carlo simulations. The package,
as of this writing, is not particularly optimized for speed, but
to allow for the user to easily construct arbitrary MCMC schemes.

\subsection{Univariate and multivariate sampling}
We wish to use a metropolis sampler to draw from a distribution with
density \[ f(x) \propto \frac{\sin(x)}{x} \cdot 1_{(0,\pi)}(x). \] We
can use a uniform distribution on $(-1,1)$ to propose distances to
jump from the current location.

Since the proposal distribution is symmetric, this is enough
information to implement a Metropolis updater. We must implement a
log-density function\footnote{Actually, this function need only be
  known up to the normalizing constant.} that defines the
distribution. We must also implement a proposal generation function.

\begin{knitrout}
\definecolor{shadecolor}{rgb}{0.969, 0.969, 0.969}\color{fgcolor}\begin{kframe}
\begin{alltt}
\hlstd{ln.d} \hlkwb{<-} \hlkwa{function}\hlstd{(}\hlkwc{x}\hlstd{)} \hlkwd{log}\hlstd{(}\hlkwd{sin}\hlstd{(x)}\hlopt{/}\hlstd{x} \hlopt{*} \hlstd{(}\hlnum{0} \hlopt{<} \hlstd{x)} \hlopt{*} \hlstd{(x} \hlopt{<} \hlstd{pi))}
\hlstd{propose} \hlkwb{<-} \hlkwa{function}\hlstd{(}\hlkwc{x}\hlstd{) x} \hlopt{+} \hlkwd{runif}\hlstd{(}\hlkwd{length}\hlstd{(x),} \hlopt{-}\hlnum{1}\hlstd{,} \hlnum{1}\hlstd{)}
\end{alltt}
\end{kframe}
\end{knitrout}


The {\tt metropolis} function accepts the functions that we have
defined, and returns a new function. The returned function is an
implementation of the Metropolis updating scheme you have
defined.\footnote{If your proposal generator is not symmetric, then
  you must also supply a log-density for the proposal mechanism, and
  in this case the function returned will implement
  Metropolis-Hastings.}

\begin{knitrout}
\definecolor{shadecolor}{rgb}{0.969, 0.969, 0.969}\color{fgcolor}\begin{kframe}
\begin{alltt}
\hlstd{updater} \hlkwb{<-} \hlkwd{metropolis}\hlstd{(ln.d, propose)}
\end{alltt}
\end{kframe}
\end{knitrout}


A Markov chain is formed by iteratively calling the updating function
starting with some initial value. The {\tt iterate} function forms a
{\it chain object}, which is simply a list of states that the chain
realized. If the state object is of a few particularly simple (but
widely applicable) forms then there are convenient methods defined for
several common R functions.
\begin{knitrout}
\definecolor{shadecolor}{rgb}{0.969, 0.969, 0.969}\color{fgcolor}\begin{kframe}
\begin{alltt}
\hlstd{chain} \hlkwb{<-} \hlkwd{iterate}\hlstd{(}\hlkwc{n} \hlstd{=} \hlnum{10000}\hlstd{,} \hlkwc{fn} \hlstd{= updater,} \hlkwc{init} \hlstd{=} \hlnum{1}\hlstd{)}
\hlkwd{summary}\hlstd{(chain,} \hlkwc{discard} \hlstd{=} \hlnum{57}\hlstd{)}
\end{alltt}
\begin{verbatim}
## Discarding first 57 states.
##    mean      se    2.5%   97.5% 
## 1.07001 0.73139 0.03456 2.63735
\end{verbatim}
\end{kframe}
\end{knitrout}


There is a method for the summary function that reports summary
statistics for the chain. A method for the hist function will create a
histogram from the chain object. By default, these methods discard the
first 10\% of values from the chain as a burn-in, but this may be
altered if desired by the user.
\begin{knitrout}
\definecolor{shadecolor}{rgb}{0.969, 0.969, 0.969}\color{fgcolor}\begin{kframe}
\begin{alltt}
\hlkwd{hist}\hlstd{(chain,} \hlkwc{breaks} \hlstd{=} \hlstr{"fd"}\hlstd{)}
\end{alltt}
\begin{verbatim}
## Discarding first 1000 states.
\end{verbatim}
\end{kframe}

{\centering \includegraphics[width=\maxwidth]{figure/chain1hist} 

}



\end{knitrout}


Of course, multivariate distributions may also be sampled. We need
only define an appropriate multivariate log-density function. A quick
review of the {\tt propose} function we defined for the univariate
sampler confirms that it will also work in the case when it is passed
a vector. (Due to the {\tt length(x)} rather than {\tt 1}.) We pass these functions to
{\tt metropolis} and obtain our updating function, which is used with
{\tt iterate} to form the Markov chain.
\begin{knitrout}
\definecolor{shadecolor}{rgb}{0.969, 0.969, 0.969}\color{fgcolor}\begin{kframe}
\begin{alltt}
\hlstd{mvtarget} \hlkwb{<-} \hlkwa{function}\hlstd{(}\hlkwc{x}\hlstd{)} \hlkwd{log}\hlstd{(}\hlkwd{prod}\hlstd{(}\hlkwd{sin}\hlstd{(}\hlkwd{prod}\hlstd{(x))}\hlopt{/}\hlstd{x)} \hlopt{*} \hlkwd{all}\hlstd{(x} \hlopt{>} \hlnum{0}\hlstd{, x} \hlopt{<} \hlstd{pi,} \hlkwd{prod}\hlstd{(x)} \hlopt{<}
    \hlstd{pi))}
\hlstd{mvupdate} \hlkwb{<-} \hlkwd{metropolis}\hlstd{(mvtarget, propose)}
\hlstd{chain2} \hlkwb{<-} \hlkwd{iterate}\hlstd{(}\hlnum{10000}\hlstd{, mvupdate,} \hlkwd{cbind}\hlstd{(}\hlkwc{x} \hlstd{=} \hlnum{1}\hlstd{,} \hlkwc{y} \hlstd{=} \hlnum{1}\hlstd{))}
\hlkwd{summary}\hlstd{(chain2)}
\end{alltt}
\begin{verbatim}
## Discarding first 1000 states.
##            x      y
## mean  1.3051 1.2055
## se    0.8184 0.7974
## 2.5%  0.1707 0.1665
## 97.5% 2.9813 2.9978
\end{verbatim}
\begin{alltt}
\hlkwd{plot}\hlstd{(chain2)}
\end{alltt}
\end{kframe}

{\centering \includegraphics[width=\maxwidth]{figure/mvt1} 

}


\begin{kframe}\begin{alltt}
\hlkwd{plot}\hlstd{(}\hlkwd{t}\hlstd{(}\hlkwd{simplify2array}\hlstd{(chain2)[}\hlnum{1}\hlstd{, , ]))}
\end{alltt}
\end{kframe}

{\centering \includegraphics[width=\maxwidth]{figure/mvt2} 

}



\end{knitrout}


\subsection{Multiple chains in parallel}
If the target density returns a vector instead of a scalar, we can run
multiple chains in parallel. Our proposal updater should update the
entire population. If a state is a matrix (as created here by {\tt
  rbind}), then plot ({\it et. al.}) assumes multiple parallel
chains.\footnote{The whole issue of what the ``default'' structure of
  a state is flexible at this point, and I haven't given much
  thought yet to what the optimal setup may be. On the todo list.}
\begin{knitrout}
\definecolor{shadecolor}{rgb}{0.969, 0.969, 0.969}\color{fgcolor}\begin{kframe}
\begin{alltt}
\hlstd{chain3} \hlkwb{<-} \hlkwd{iterate}\hlstd{(}\hlnum{1000}\hlstd{, updater,} \hlkwc{init} \hlstd{=} \hlkwd{rbind}\hlstd{(}\hlnum{0.1}\hlstd{,} \hlnum{0.5}\hlstd{,} \hlnum{2.5}\hlstd{,} \hlnum{3.1}\hlstd{))}
\hlkwd{plot}\hlstd{(chain3)}
\end{alltt}
\end{kframe}

{\centering \includegraphics[width=\maxwidth]{figure/parallel} 

}



\end{knitrout}


Multiple chains of multivariate samples are also possible, as long as
a logical vector of the same length as that returned by the density
function will ``correctly'' subset individuals from the population
state object. \footnote{Thus, a population can be a vector, a matrix with
individuals in rows, or a list. Some convenience methods such as
print, summary, plot, hist, etc., assume a population is a matrix, but
this is not strictly required.}
\begin{knitrout}
\definecolor{shadecolor}{rgb}{0.969, 0.969, 0.969}\color{fgcolor}\begin{kframe}
\begin{alltt}
\hlstd{mvt2} \hlkwb{<-} \hlkwa{function}\hlstd{(}\hlkwc{x}\hlstd{)} \hlkwd{apply}\hlstd{(x,} \hlnum{1}\hlstd{, mvtarget)}
\hlstd{mvup2} \hlkwb{<-} \hlkwd{metropolis}\hlstd{(mvt2, propose)}
\hlstd{init} \hlkwb{<-} \hlkwd{rbind}\hlstd{(}\hlkwd{c}\hlstd{(}\hlkwc{x} \hlstd{=} \hlnum{1}\hlstd{,} \hlkwc{y} \hlstd{=} \hlnum{3}\hlstd{),} \hlkwd{c}\hlstd{(}\hlnum{0.1}\hlstd{,} \hlnum{0.5}\hlstd{),} \hlkwd{c}\hlstd{(}\hlnum{1}\hlstd{,} \hlnum{1}\hlstd{))}
\hlstd{chain4} \hlkwb{<-} \hlkwd{iterate}\hlstd{(}\hlnum{10000}\hlstd{, mvup2, init)}
\hlkwd{plot}\hlstd{(chain4)}
\end{alltt}
\end{kframe}

{\centering \includegraphics[width=\maxwidth]{figure/multi-multi} 

}



\end{knitrout}


\subsection{Gibbs sampling}
\label{sec:gibbs-sampling}


A gibbs updater calls several updating functions sequentially.  The
following example problem is given in Casella and George
(1992).\footnote{I actually found this at
  \url{http://web.mit.edu/~wingated/www/introductions/mcmc-gibbs-intro.pdf}.}
We wish to sample from a joint distribution
\[p(x, y) = {n \choose x} y^{x+\alpha-1} (1-y)^{n-x+\beta-1}.\]
The conditional distributions are 
$x|y \sim \text{Bin}(n, y)$, and $y |x \sim \text{Beta}(x +
\alpha, n-x+\beta)$. We let $n=100$, and $\alpha=\beta=2$ in this example.
\begin{knitrout}
\definecolor{shadecolor}{rgb}{0.969, 0.969, 0.969}\color{fgcolor}\begin{kframe}
\begin{alltt}
\hlcom{## state = cbind(x,y)}
\hlstd{x.y} \hlkwb{<-} \hlkwa{function}\hlstd{(}\hlkwc{state}\hlstd{) \{}
    \hlstd{state[,} \hlnum{1}\hlstd{]} \hlkwb{<-} \hlkwd{rbinom}\hlstd{(}\hlkwd{nrow}\hlstd{(state),} \hlnum{100}\hlstd{, state[,} \hlnum{2}\hlstd{])}
    \hlstd{state}
\hlstd{\}}
\hlstd{y.x} \hlkwb{<-} \hlkwa{function}\hlstd{(}\hlkwc{state}\hlstd{) \{}
    \hlstd{state[,} \hlnum{2}\hlstd{]} \hlkwb{<-} \hlkwd{rbeta}\hlstd{(}\hlkwd{nrow}\hlstd{(state), state[,} \hlnum{1}\hlstd{]} \hlopt{+} \hlnum{2}\hlstd{,} \hlnum{100} \hlopt{-} \hlstd{state[,} \hlnum{1}\hlstd{]} \hlopt{+} \hlnum{2}\hlstd{)}
    \hlstd{state}
\hlstd{\}}
\end{alltt}
\end{kframe}
\end{knitrout}

The gibbs function is used to create a function which acts as the
Gibbs updater.

\begin{knitrout}
\definecolor{shadecolor}{rgb}{0.969, 0.969, 0.969}\color{fgcolor}\begin{kframe}
\begin{alltt}
\hlstd{init} \hlkwb{<-} \hlkwd{cbind}\hlstd{(}\hlkwc{x} \hlstd{=} \hlnum{1}\hlopt{:}\hlnum{10}\hlstd{,} \hlkwc{y} \hlstd{=} \hlkwd{seq}\hlstd{(}\hlnum{0.1}\hlstd{,} \hlnum{0.9}\hlstd{,} \hlkwc{length.out} \hlstd{=} \hlnum{10}\hlstd{))}
\hlstd{gibbs.updater} \hlkwb{<-} \hlkwd{gibbs}\hlstd{(x.y, y.x)}
\hlstd{chain5} \hlkwb{<-} \hlkwd{iterate}\hlstd{(}\hlnum{5000}\hlstd{, gibbs.updater, init)}
\hlkwd{plot}\hlstd{(}\hlkwd{t}\hlstd{(}\hlkwd{simplify2array}\hlstd{(chain5)[}\hlnum{1}\hlstd{, , ]))}
\end{alltt}
\end{kframe}

{\centering \includegraphics[width=0.7\textwidth]{figure/gibbs} 

}



\end{knitrout}


Of course, you can use a metropolis updater in place of a full
conditional sampler as we did above. In the extreme case, the one-
parameter-at-a-time metropolis updating scheme can be implemented
using {\tt gibbs}.

\subsection{Parallel tempering}
We want to sample from a distribution with separated modes.  First we
set up the target distribution function. This function should take a
single individual and return the density of that individual.
\begin{knitrout}
\definecolor{shadecolor}{rgb}{0.969, 0.969, 0.969}\color{fgcolor}\begin{kframe}
\begin{alltt}
\hlstd{f} \hlkwb{<-} \hlkwa{function}\hlstd{(}\hlkwc{state}\hlstd{)} \hlkwd{log}\hlstd{(}\hlkwd{dnorm}\hlstd{(state)} \hlopt{+} \hlkwd{dnorm}\hlstd{(state,} \hlnum{10}\hlstd{))}
\end{alltt}
\end{kframe}
\end{knitrout}


Now we can set up the parallel tempering updater. We define a
population to have individuals in rows and multiple variates. First we
use the {\tt heat} function to create a function which will evaluate
heated density values for each individual in the population. The {\tt
  mutate} function does a metropolis update of each individual in the
population using the supplied proposal function as described
previously, then it attempts to exchange individuals between
temperatures using an appropriate metropolis scheme.

\begin{knitrout}
\definecolor{shadecolor}{rgb}{0.969, 0.969, 0.969}\color{fgcolor}\begin{kframe}
\begin{alltt}
\hlstd{temps} \hlkwb{<-} \hlkwd{c}\hlstd{(}\hlnum{1}\hlstd{,} \hlnum{2}\hlstd{,} \hlnum{4}\hlstd{,} \hlnum{8}\hlstd{,} \hlnum{15}\hlstd{)}
\hlstd{hot.dens} \hlkwb{<-} \hlkwd{heat}\hlstd{(f, temps)}
\hlstd{updater} \hlkwb{<-} \hlkwd{mutate}\hlstd{(hot.dens, propose)}
\hlstd{init} \hlkwb{<-} \hlkwd{rbind}\hlstd{(}\hlopt{-}\hlnum{3}\hlstd{,} \hlnum{0}\hlstd{,} \hlnum{5}\hlstd{,} \hlnum{10}\hlstd{,} \hlnum{15}\hlstd{)}
\hlstd{gch} \hlkwb{<-} \hlkwd{iterate}\hlstd{(}\hlnum{10000}\hlstd{, updater, init)}
\hlkwd{plot}\hlstd{(gch)}
\end{alltt}
\end{kframe}

{\centering \includegraphics[width=0.8\textwidth]{figure/tempering} 

}



\end{knitrout}

\begin{knitrout}
\definecolor{shadecolor}{rgb}{0.969, 0.969, 0.969}\color{fgcolor}\begin{kframe}
\begin{alltt}
\hlstd{xx} \hlkwb{<-} \hlkwd{prune}\hlstd{(gch,} \hlnum{1}\hlstd{,} \hlnum{1}\hlstd{)}  \hlcom{#extract chain with correct distribution}
\hlkwd{hist}\hlstd{(xx,} \hlkwc{freq} \hlstd{=} \hlnum{FALSE}\hlstd{,} \hlkwc{breaks} \hlstd{=} \hlstr{"fd"}\hlstd{)}
\hlkwd{curve}\hlstd{((}\hlkwd{dnorm}\hlstd{(x)} \hlopt{+} \hlkwd{dnorm}\hlstd{(x,} \hlnum{10}\hlstd{))}\hlopt{/}\hlnum{2}\hlstd{,} \hlkwc{add} \hlstd{=} \hlnum{TRUE}\hlstd{)}
\end{alltt}
\end{kframe}

{\centering \includegraphics[width=\maxwidth]{figure/unnamed-chunk-3} 

}



\end{knitrout}


\subsection{Evolutionary MC}

Here we go with the real-deal. Twenty part trivariate-gaussian mixure.
The target function is a part of the evolMC package. 

\begin{knitrout}
\definecolor{shadecolor}{rgb}{0.969, 0.969, 0.969}\color{fgcolor}\begin{kframe}
\begin{alltt}
\hlstd{means} \hlkwb{<-} \hlkwd{t}\hlstd{(}\hlkwd{replicate}\hlstd{(}\hlnum{20}\hlstd{,} \hlkwd{runif}\hlstd{(}\hlnum{3}\hlstd{,} \hlopt{-}\hlnum{10}\hlstd{,} \hlnum{10}\hlstd{)))}
\hlstd{dmix} \hlkwb{<-} \hlkwa{function}\hlstd{(}\hlkwc{x}\hlstd{)} \hlkwd{target}\hlstd{(x, means)}
\hlstd{ladder} \hlkwb{<-} \hlkwd{heat}\hlstd{(dmix, temps)}
\hlstd{e} \hlkwb{<-} \hlkwd{seq}\hlstd{(}\hlnum{0.2}\hlstd{,} \hlnum{3}\hlstd{,} \hlkwc{length} \hlstd{=} \hlnum{5}\hlstd{)}
\hlstd{runif3} \hlkwb{<-} \hlkwa{function}\hlstd{(}\hlkwc{x}\hlstd{) x} \hlopt{+} \hlkwd{runif}\hlstd{(}\hlkwd{length}\hlstd{(x),} \hlopt{-}\hlstd{e, e)}
\hlstd{runif4} \hlkwb{<-} \hlkwa{function}\hlstd{(}\hlkwc{x}\hlstd{) x} \hlopt{+} \hlkwd{rnorm}\hlstd{(}\hlkwd{length}\hlstd{(x),} \hlnum{0}\hlstd{,} \hlnum{0.25} \hlopt{*} \hlstd{temps)}
\hlstd{evolve} \hlkwb{<-} \hlkwd{gibbs}\hlstd{(}\hlkwd{reproduce}\hlstd{(ladder),} \hlkwd{mutate}\hlstd{(ladder, runif4),} \hlkwc{p} \hlstd{=} \hlkwd{c}\hlstd{(}\hlnum{1}\hlstd{,} \hlnum{1}\hlstd{))}
\hlstd{ptemp} \hlkwb{<-} \hlkwd{mutate}\hlstd{(ladder, runif4)}
\hlstd{wt.evolve} \hlkwb{<-} \hlkwd{gibbs}\hlstd{(}\hlkwd{weighted.reproduce}\hlstd{(ladder),} \hlkwd{mutate}\hlstd{(ladder, runif4),} \hlkwc{p} \hlstd{=} \hlkwd{c}\hlstd{(}\hlnum{1}\hlstd{,}
    \hlnum{1}\hlstd{))}

\hlstd{init} \hlkwb{<-} \hlkwd{matrix}\hlstd{(}\hlnum{0}\hlstd{,} \hlnum{5}\hlstd{,} \hlnum{3}\hlstd{)}
\hlcom{# invisible(runif(100)) lineage <- iterate(2000, evolve, init, bar=1)}
\hlcom{# pt.lineage <- iterate(2000, ptemp, init, bar=1)}
\hlstd{wt.lineage} \hlkwb{<-} \hlkwd{iterate}\hlstd{(}\hlnum{2000}\hlstd{, wt.evolve, init,} \hlkwc{bar} \hlstd{=} \hlnum{1}\hlstd{)}
\end{alltt}
\begin{verbatim}
## ===========================================================================
\end{verbatim}
\end{kframe}
\end{knitrout}


\begin{knitrout}
\definecolor{shadecolor}{rgb}{0.969, 0.969, 0.969}\color{fgcolor}\begin{kframe}
\begin{alltt}
\hlkwd{plot}\hlstd{(lineage)}
\end{alltt}
\end{kframe}

{\centering \includegraphics[width=\maxwidth]{figure/unnamed-chunk-41} 

}


\begin{kframe}\begin{alltt}
\hlstd{m} \hlkwb{<-} \hlkwd{aperm}\hlstd{(}\hlkwd{simplify2array}\hlstd{(wt.lineage))}
\hlcom{## wt.m <- aperm(simplify2array(wt.lineage)) pt.m <-}
\hlcom{## aperm(simplify2array(pt.lineage))}
\hlkwd{library}\hlstd{(scatterplot3d)}
\hlstd{pp} \hlkwb{<-} \hlkwd{scatterplot3d}\hlstd{(means)}
\hlcom{# invisible(sapply(5:1,function(i) pp$points3d(m[,,i],col=i+1)))}
\hlstd{pp}\hlopt{$}\hlkwd{points3d}\hlstd{(m[, ,} \hlnum{1}\hlstd{],} \hlkwc{col} \hlstd{=} \hlnum{2}\hlstd{)}
\hlcom{## pp$points3d(wt.m[,,1],col=3) pp$points3d(pt.m[,,1],col=3)}
\hlcom{## pp$points3d(m[,,1],col=2)}

\hlstd{pp}\hlopt{$}\hlkwd{points3d}\hlstd{(means,} \hlkwc{col} \hlstd{=} \hlnum{1}\hlstd{,} \hlkwc{pch} \hlstd{=} \hlnum{19}\hlstd{)}
\end{alltt}
\end{kframe}

{\centering \includegraphics[width=\maxwidth]{figure/unnamed-chunk-42} 

}


\begin{kframe}\begin{alltt}
\hlkwd{pairs}\hlstd{(m[, ,} \hlnum{1}\hlstd{])}
\end{alltt}
\end{kframe}

{\centering \includegraphics[width=\maxwidth]{figure/unnamed-chunk-43} 

}


\begin{kframe}\begin{alltt}
\hlcom{## library(rgl) plot3d(m[,,1]) plot3d(means,pch=19,size=15)}
\hlcom{## sapply(1:5,function(i) points3d(m[,,i],col=i+1))}
\end{alltt}
\end{kframe}
\end{knitrout}

\end{document}
