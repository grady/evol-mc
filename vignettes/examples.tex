\documentclass{article}\usepackage[]{graphicx}\usepackage[]{color}
%% maxwidth is the original width if it is less than linewidth
%% otherwise use linewidth (to make sure the graphics do not exceed the margin)
\makeatletter
\def\maxwidth{ %
  \ifdim\Gin@nat@width>\linewidth
    \linewidth
  \else
    \Gin@nat@width
  \fi
}
\makeatother

\definecolor{fgcolor}{rgb}{0.345, 0.345, 0.345}
\newcommand{\hlnum}[1]{\textcolor[rgb]{0.686,0.059,0.569}{#1}}%
\newcommand{\hlstr}[1]{\textcolor[rgb]{0.192,0.494,0.8}{#1}}%
\newcommand{\hlcom}[1]{\textcolor[rgb]{0.678,0.584,0.686}{\textit{#1}}}%
\newcommand{\hlopt}[1]{\textcolor[rgb]{0,0,0}{#1}}%
\newcommand{\hlstd}[1]{\textcolor[rgb]{0.345,0.345,0.345}{#1}}%
\newcommand{\hlkwa}[1]{\textcolor[rgb]{0.161,0.373,0.58}{\textbf{#1}}}%
\newcommand{\hlkwb}[1]{\textcolor[rgb]{0.69,0.353,0.396}{#1}}%
\newcommand{\hlkwc}[1]{\textcolor[rgb]{0.333,0.667,0.333}{#1}}%
\newcommand{\hlkwd}[1]{\textcolor[rgb]{0.737,0.353,0.396}{\textbf{#1}}}%

\usepackage{framed}
\makeatletter
\newenvironment{kframe}{%
 \def\at@end@of@kframe{}%
 \ifinner\ifhmode%
  \def\at@end@of@kframe{\end{minipage}}%
  \begin{minipage}{\columnwidth}%
 \fi\fi%
 \def\FrameCommand##1{\hskip\@totalleftmargin \hskip-\fboxsep
 \colorbox{shadecolor}{##1}\hskip-\fboxsep
     % There is no \\@totalrightmargin, so:
     \hskip-\linewidth \hskip-\@totalleftmargin \hskip\columnwidth}%
 \MakeFramed {\advance\hsize-\width
   \@totalleftmargin\z@ \linewidth\hsize
   \@setminipage}}%
 {\par\unskip\endMakeFramed%
 \at@end@of@kframe}
\makeatother

\definecolor{shadecolor}{rgb}{.97, .97, .97}
\definecolor{messagecolor}{rgb}{0, 0, 0}
\definecolor{warningcolor}{rgb}{1, 0, 1}
\definecolor{errorcolor}{rgb}{1, 0, 0}
\newenvironment{knitrout}{}{} % an empty environment to be redefined in TeX

\usepackage{alltt}

\title{evolMC demo}
\author{Grady Weyenberg}
\IfFileExists{upquote.sty}{\usepackage{upquote}}{}
\begin{document}
\maketitle

evolMC is a framework for doing Monte-Carlo simulations.

We wish to use a metropolis sampler to draw from a distribution with
density
\[ f(x) \propto \frac{sin(x)}{x} \cdot 1_{(0,\pi)}(x). \]
\begin{knitrout}
\definecolor{shadecolor}{rgb}{0.969, 0.969, 0.969}\color{fgcolor}\begin{kframe}
\begin{alltt}
\hlkwd{library}\hlstd{(evolMC)}
\hlstd{fn} \hlkwb{<-} \hlkwa{function}\hlstd{(}\hlkwc{x}\hlstd{)} \hlkwd{sin}\hlstd{(x)}\hlopt{/}\hlstd{x} \hlopt{*} \hlstd{(}\hlnum{0} \hlopt{<} \hlstd{x)} \hlopt{*} \hlstd{(x} \hlopt{<} \hlstd{pi)}
\end{alltt}
\end{kframe}
\end{knitrout}


We can use a uniform distribution on $(-1,1)$ to propose distances to
jump from the current location.
\begin{knitrout}
\definecolor{shadecolor}{rgb}{0.969, 0.969, 0.969}\color{fgcolor}\begin{kframe}
\begin{alltt}
\hlstd{propose} \hlkwb{<-} \hlkwa{function}\hlstd{(}\hlkwc{x}\hlstd{) x} \hlopt{+} \hlkwd{runif}\hlstd{(}\hlkwd{length}\hlstd{(x),} \hlopt{-}\hlnum{1}\hlstd{,} \hlnum{1}\hlstd{)}
\end{alltt}
\end{kframe}
\end{knitrout}


Since the proposal distribution is symmetric, this is enough
information to implement a Metropolis updater.
\begin{knitrout}
\definecolor{shadecolor}{rgb}{0.969, 0.969, 0.969}\color{fgcolor}\begin{kframe}
\begin{alltt}
\hlstd{updater} \hlkwb{<-} \hlkwd{metropolis}\hlstd{(fn, propose)}
\end{alltt}
\end{kframe}
\end{knitrout}


A Markov chain is formed by iteratively calling the updating function
starting with some initial value.
\begin{knitrout}
\definecolor{shadecolor}{rgb}{0.969, 0.969, 0.969}\color{fgcolor}\begin{kframe}
\begin{alltt}
\hlstd{chain} \hlkwb{<-} \hlkwd{iterate}\hlstd{(}\hlnum{10000}\hlstd{, updater,} \hlkwc{init} \hlstd{=} \hlkwd{matrix}\hlstd{(}\hlnum{1}\hlstd{))}
\hlkwd{summary}\hlstd{(chain)}
\end{alltt}
\begin{verbatim}
## Discarding first 1000 states.
##   mean     se   2.5%  97.5% 
## 1.0606 0.7365 0.0399 2.6534
\end{verbatim}
\end{kframe}
\end{knitrout}


\begin{knitrout}
\definecolor{shadecolor}{rgb}{0.969, 0.969, 0.969}\color{fgcolor}\begin{kframe}
\begin{alltt}
\hlkwd{hist}\hlstd{(chain,} \hlkwc{breaks} \hlstd{=} \hlstr{"fd"}\hlstd{)}
\end{alltt}
\begin{verbatim}
## Discarding first 1000 states.
\end{verbatim}
\end{kframe}
\includegraphics[width=\maxwidth]{figure/unnamed-chunk-5} 

\end{knitrout}


Of course, multivariate distributions may also be sampled. 
\begin{knitrout}
\definecolor{shadecolor}{rgb}{0.969, 0.969, 0.969}\color{fgcolor}\begin{kframe}
\begin{alltt}
\hlstd{mvtarget} \hlkwb{<-} \hlkwa{function}\hlstd{(}\hlkwc{x}\hlstd{)} \hlkwd{prod}\hlstd{(}\hlkwd{sin}\hlstd{(x)}\hlopt{/}\hlstd{x)} \hlopt{*} \hlkwd{all}\hlstd{(x} \hlopt{>} \hlnum{0}\hlstd{)} \hlopt{*} \hlkwd{all}\hlstd{(x} \hlopt{<} \hlstd{pi)}
\hlstd{mvupdate} \hlkwb{<-} \hlkwd{metropolis}\hlstd{(mvtarget, propose)}
\hlstd{chain2} \hlkwb{<-} \hlkwd{iterate}\hlstd{(}\hlnum{1000}\hlstd{, mvupdate,} \hlkwd{cbind}\hlstd{(}\hlkwc{x} \hlstd{=} \hlnum{1}\hlstd{,} \hlkwc{y} \hlstd{=} \hlnum{2}\hlstd{))}
\hlkwd{summary}\hlstd{(chain2)}
\end{alltt}
\begin{verbatim}
## Discarding first 100 states.
##             x      y
## mean  0.92918 0.9995
## se    0.67649 0.7103
## 2.5%  0.04376 0.0510
## 97.5% 2.43684 2.5511
\end{verbatim}
\begin{alltt}
\hlkwd{plot}\hlstd{(chain2)}
\end{alltt}
\end{kframe}
\includegraphics[width=\maxwidth]{figure/unnamed-chunk-6} 

\end{knitrout}


If the target density and proposal generator are ``carefully
selected'' then multiple chains can be run in parallel.  
\begin{knitrout}
\definecolor{shadecolor}{rgb}{0.969, 0.969, 0.969}\color{fgcolor}\begin{kframe}
\begin{alltt}
\hlstd{chain2} \hlkwb{<-} \hlkwd{iterate}\hlstd{(}\hlnum{100}\hlstd{, updater,} \hlkwc{init} \hlstd{=} \hlkwd{rbind}\hlstd{(}\hlnum{0.1}\hlstd{,} \hlnum{0.5}\hlstd{,} \hlnum{2.5}\hlstd{,} \hlnum{3.1}\hlstd{))}
\hlkwd{plot}\hlstd{(chain2)}
\end{alltt}
\end{kframe}
\includegraphics[width=\maxwidth]{figure/unnamed-chunk-7} 

\end{knitrout}


Multiple chains of multivariate samples are also possible
\begin{knitrout}
\definecolor{shadecolor}{rgb}{0.969, 0.969, 0.969}\color{fgcolor}\begin{kframe}
\begin{alltt}
\hlstd{mvt2} \hlkwb{<-} \hlkwa{function}\hlstd{(}\hlkwc{x}\hlstd{)} \hlkwd{apply}\hlstd{(x,} \hlnum{1}\hlstd{, mvtarget)}
\hlstd{mvup2} \hlkwb{<-} \hlkwd{metropolis}\hlstd{(mvt2, propose)}
\hlstd{init} \hlkwb{<-} \hlkwd{rbind}\hlstd{(}\hlkwd{c}\hlstd{(}\hlkwc{x} \hlstd{=} \hlnum{1}\hlstd{,} \hlkwc{y} \hlstd{=} \hlnum{2}\hlstd{),} \hlkwd{c}\hlstd{(}\hlnum{0.1}\hlstd{,} \hlnum{0.5}\hlstd{),} \hlkwd{c}\hlstd{(}\hlnum{3}\hlstd{,} \hlnum{2.5}\hlstd{))}
\hlstd{chain3} \hlkwb{<-} \hlkwd{iterate}\hlstd{(}\hlnum{1000}\hlstd{, mvup2, init)}
\hlkwd{plot}\hlstd{(chain3)}
\end{alltt}
\end{kframe}
\includegraphics[width=\maxwidth]{figure/unnamed-chunk-8} 

\end{knitrout}


The chain methods, such as summary, plot, and hist, assume that a
state is a matrix with individuals in rows, and multiple variables in
columns, however this is not required by the functions which generate
the chain. The main requirement is that the following expression is
valid and that the resulting set of logical values can be used to
``correctly'' subset the state object. (Where ``target'' is the
density function, and ``state'' is the current state of the chain.)

\begin{knitrout}
\definecolor{shadecolor}{rgb}{0.969, 0.969, 0.969}\color{fgcolor}\begin{kframe}
\begin{alltt}
\hlkwd{target}\hlstd{(state)} \hlopt{<} \hlnum{1}
\end{alltt}
\end{kframe}
\end{knitrout}

\end{document}
