\documentclass{article}\usepackage[]{graphicx}\usepackage[]{color}
%% maxwidth is the original width if it is less than linewidth
%% otherwise use linewidth (to make sure the graphics do not exceed the margin)
\makeatletter
\def\maxwidth{ %
  \ifdim\Gin@nat@width>\linewidth
    \linewidth
  \else
    \Gin@nat@width
  \fi
}
\makeatother

\definecolor{fgcolor}{rgb}{0.345, 0.345, 0.345}
\newcommand{\hlnum}[1]{\textcolor[rgb]{0.686,0.059,0.569}{#1}}%
\newcommand{\hlstr}[1]{\textcolor[rgb]{0.192,0.494,0.8}{#1}}%
\newcommand{\hlcom}[1]{\textcolor[rgb]{0.678,0.584,0.686}{\textit{#1}}}%
\newcommand{\hlopt}[1]{\textcolor[rgb]{0,0,0}{#1}}%
\newcommand{\hlstd}[1]{\textcolor[rgb]{0.345,0.345,0.345}{#1}}%
\newcommand{\hlkwa}[1]{\textcolor[rgb]{0.161,0.373,0.58}{\textbf{#1}}}%
\newcommand{\hlkwb}[1]{\textcolor[rgb]{0.69,0.353,0.396}{#1}}%
\newcommand{\hlkwc}[1]{\textcolor[rgb]{0.333,0.667,0.333}{#1}}%
\newcommand{\hlkwd}[1]{\textcolor[rgb]{0.737,0.353,0.396}{\textbf{#1}}}%

\usepackage{framed}
\makeatletter
\newenvironment{kframe}{%
 \def\at@end@of@kframe{}%
 \ifinner\ifhmode%
  \def\at@end@of@kframe{\end{minipage}}%
  \begin{minipage}{\columnwidth}%
 \fi\fi%
 \def\FrameCommand##1{\hskip\@totalleftmargin \hskip-\fboxsep
 \colorbox{shadecolor}{##1}\hskip-\fboxsep
     % There is no \\@totalrightmargin, so:
     \hskip-\linewidth \hskip-\@totalleftmargin \hskip\columnwidth}%
 \MakeFramed {\advance\hsize-\width
   \@totalleftmargin\z@ \linewidth\hsize
   \@setminipage}}%
 {\par\unskip\endMakeFramed%
 \at@end@of@kframe}
\makeatother

\definecolor{shadecolor}{rgb}{.97, .97, .97}
\definecolor{messagecolor}{rgb}{0, 0, 0}
\definecolor{warningcolor}{rgb}{1, 0, 1}
\definecolor{errorcolor}{rgb}{1, 0, 0}
\newenvironment{knitrout}{}{} % an empty environment to be redefined in TeX

\usepackage{alltt}
\usepackage{amssymb}
\usepackage{natbib}
%\usepackage{fullpage}
%\VignetteIndexEntry{evolMC demo} 
%\VignetteDepends{evolMC,knitr} 
%\VignetteEngine{knitr::knitr}
\newcommand{\grady}[1]{\marginpar{#1--Grady}}
\newcommand{\bx}{\mathbf x}
\newcommand{\bt}{\mathbf t}
\newcommand{\X}{\mathcal X}




\title{evolMC: a package for Monte-Carlo simulation}
\author{W Burchett \and E Roualdes \and G Weyenberg}
\IfFileExists{upquote.sty}{\usepackage{upquote}}{}


\begin{document}

\maketitle


\begin{abstract}
  Sampling from multi-modal target distributions proves difficult for
  basic MCMC methods.  Borrowing from the evolutionary based
  algorithms of optimization theory, \cite{liang2000evolutionary}
  introduced evolutionary Monte Carlo (EMC).  EMC begins with parallel
  tempering and then adds to it a step in which sequences
  intelligently evolve by learning from each other.  A comparative
  simulation highlights the effectiveness of EMC over other sampling
  methods.
\end{abstract}

%%\section{Introduction}
\vspace{1cm}
\label{sec:introduction}

Population based MCMC methods are becoming increasingly common in
applied statistics.  Parallel tempering, for instance, starts by
simulating multiple sequences from a heated target distribution to
encourage a more thorough global search of the support space.
\cite{liang2000evolutionary} introduced evolutionary MC that builds
upon, and generalizes, parallel tempering by blending with it theory
from evolutionary aglorithms. Evolutionary Monte Carlo (EMC) is a
population based Monte Carlo method that intelligently evolves at each
iteration.

The discussion of population based Monte Carlo methods necessarily
begins by introducting some terminology.  We wish to sample from a
distribution \[f(x|t) \propto \exp\{ -H(x)/t \}\] where $t\ge 1$ is
called the \emph{temperature} and $H(x)$, called the \emph{fitness
  function}, corresponds to the negative log-density of $x$, up to a
constant. A \emph{population} $\mathbf{x}$ consists of $N$
\emph{individuals} $x_i$, $i = 1, \ldots, N$, and an associated set of
temperatures $\bt = (t_1,\ldots,t_N)$. ($x_i \in \mathbb{R}^d$ and the
$t_i$ are in decending order.) Each
individual $x_i$ is independently sampled from the distribution
$f_i(x_i) \propto f(x_i|t_i)$.  For any such distribution, the
\emph{partition function} $Z(t)= \int_\X f(x|t) dx$ is the inverse of
the normalizing constant. \grady{Citation?  Basic fact?}


The \emph{Boltzman distribution} of the population $\bx$ is the
product of the individual distributions, $f(\bx) \propto \exp \{ -\sum_{i=1}^N
H(x_i)/ t_i \}$, with $Z(\bt) := \prod_{i=1}^N Z(t_i)$ as the inverse
of the normalizing constant.  Iterations of the EMC algorithm consist
of three operations: mutation, crossover, and exchange. Each type of
operation makes a specific type of modification to the current
population, and the change is carried forward to the next generation
with a probability governed by a Metropolis-type rule.

In general terms, a mutation operation adds noise to the current
population. This can be analogous to a basic random walk type
proposal, but more elaborate methods have been
proposed.\grady{citation} A crossover operator consists of
randomly selecting two individuals and generating replacement
individuals by somehow mixing the parent values, analogous to the
Chromosomal crossovers that sometimes occur during Mieosis.  An
exchange step proposes a switch of two randomly selected individuals
without switching their associated temperatures.


In the remainder of this paper we compare the performance of the EMC
algorithm to parallel tempering and a na\"ive Metropolis sampler in
the context of a 3 dimensional distribution with 20 modes.  Section
two describes in greater detail the simulation setup.  While results
of the simulation are presented and discussed in sections
\ref{sec:results} and \ref{sec:discussion}.

\setcounter{section}{1}
\section{Methods}
\label{sec:methods}
The target distribution for our study is a 20-part mixture of
trivariate Gaussian distributions. Each part of the mixture is
$\mathcal{N}_3(\mu_i, 0.1I_3)$, and each vector $\mu_i$ is located
uniformly in $[-10,10]\times[-10,10]\times[-10,10].$ This distribution
possesses numerous modes that are separated by relatively large
regions of low probability, making it challenging for a basic
Metropolis sampler to mix between the modes.

This section should describe the simulations we did.

\section{Results}
\label{sec:results}

\subsection{Software}
\label{sec:software}


We present the R package evolMC, which is a framework for doing
Markov-Chain Monte Carlo simulations. It includes explicit support for
several MCMC algorithms, including Metropolis-Hastings, Gibbs,
parallel tempering, and the evolutionary updating methods.

The package aims to provide a general framework for constructing a
chain, leaving it up to the user to provide the specific pieces
specific to the problem. The user must first define a \emph{state
object} and then implement a number of functions which act on this
state object. In many traditional cases this state object can simply
be a vector of reals, but any R object (e.g. a phylogenetic tree
object defined by a 3rd-party package) can be used. The functions
provided by evolMC will eventually produce a \emph{chain} object,
which is simply an R list where each element is an instance of the
state object.

\subsection{Simulations}
\label{sec:simulations}


\section{Discussion}
\label{sec:discussion}


%%BIBLIOGRAPHY HERE
\bibliography{refs}
\bibliographystyle{named}

\appendix

\section{evolMC}
\label{sec:evolmc}
The evolMC package includes a vignette demonstrating a few basic use
cases, with example code. Here we briefly discuss the architectural
vision underlying the package and a few of the most important
functions.

evolMC was designed to be as flexible as possible, and hence it
operates on a quite abstract level. It is written in a very functional
style of programming; most of the important functions in evolMC accept
functions as arguments and return new functions. An important goal was
for the user to be free to use whatever object they wish to represent
a realization of the random element in question. This departs from
many existing R MCMC packages, which require vector- or
array-type structures for the random elements in question.

The main job of the user of evolMC is to define a few basic functions
which can interact in specified ways with whatever object is selected
to represent states of the chain. For example, the user will often
need to implement a function which evaluates a log-densitiy for a
given state. Using the routines provided by evolMC, one will then
manipulate these functions, eventually producing a function which
implements some desired MCMC algorithm.

The core function in evolMC is
\texttt{iterate(n,f,x)}. This function is also among the least
interesting, because it simply creates a list where the $k$-th element
is $f^{k-1}(x)$. (Function exponentiation means iterated application
here.) If the provided function {\tt f} is ``carefully selected'' then the
resulting chain might have ``desirable properties''. Of course, we are
likely interested in the case where the desirable property in
question is that the resulting list represents a sample from some
target distribution. Most other functions in the package exist to help
one in constructing a function {\tt f} with specific desirable properties.

Perhaps the most basic MCMC algorithm is the Gibbs sampler. Suppose we
want to sample from $[x,y]$. If we can sample from $[x|y]$ and
$[y|x]$, then we can implement a Gibbs sampler. To do this one need
only implement functions {\tt rxt} and {\tt ryx}, which do the
respective sampling. These
functions should accept the current state as their first argument, and
return a new state object with the relavant changes made.

Once one has these functions, the gibbs function can be used to
produce a new updating function which will implement the Gibbs
algorithm. The basic signature is {\tt gibbs(...)}. The ...  argument
allows one to provide any number of updating functions; in our case we
have two, {\tt rxt} and {\tt ryx}. The result of {\tt gibbs(rxy,ryx)}
is a new ``carefully selected'' function with signature
\texttt{function(state)} which acts as a Gibbs updater. If called
repeatedly (e.g. by iterate) it will alternate between using {\tt rxy}
and {\tt ryx} to update the state.\footnote{The sequential choice of
  updating functions is the default mode of operation for the function
  returned by {\tt gibbs}. It can also stochastically select an
  updating function according to provided weights.}

A second important algorithm is the Metropolis algorithm. The function
{\tt metropolis} helps one to construct an updating function which
implements Metropolis-Hastings. To use this function one must first implement:
the log-density function for the target distribution ({\tt ln.d}); a proposal
generator function ({\tt r.prop}); and (if required) a log-density function for the
proposal generator ({\tt ln.d.prop}). A call {\tt metropolis(ln.d,
  r.prop, ln.d.prop)} will return a ``carefully selected'' function
(with signature {\tt function(state)}), which implements the
Metropolis-Hastings scheme you have defined.

These three basic building blocks can be used to construct a suprising
number of more advanced MCMC algorithms, and indeed many of the more
advanced agorithms provided by evolMC are constructed using only these pieces.
\end{document}
